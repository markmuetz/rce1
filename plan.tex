\documentclass{article}

% Some useful packages.
\usepackage{amsmath}
\usepackage{siunitx}
\usepackage{graphicx}
\usepackage{verbatim}
\usepackage{mhchem}
\usepackage{textcomp}
\usepackage{courier}
\usepackage{listings}

% Reduces margins substantially.
\usepackage{geometry}
\newgeometry{margin=2.5cm}

% Allows headers and footers.
\usepackage{fancyhdr}
\pagestyle{fancy}
% Get rid of annoying line under header.
\renewcommand{\headrulewidth}{0pt}

\newcommand{\ts}{\textsuperscript}

\lhead{}
\chead{}
\rhead{}

\lstset{
    basicstyle=\ttfamily,
}

\begin{document}

\section*{16/03/2016: RCE Investigation of Convective Cloud Spectrum}

\subsection*{Goals}

\begin{itemize}
    \item To learn how to pose meaningful scientific questions and set about answering them.
    \item To get experience in running the Idealized UM and analysing its output.
\end{itemize}

\subsection*{Questions to be addressed:}

\begin{enumerate}
    \item How resolution affects cloud spectrum statistics.
    \item How closely parameterized models match Cloud Resolving Models (CRMs).
    \item What happens when convection parameterization is run at CRM resolution.
\end{enumerate}

When addressing question 1, I will need to use diagnostic tools for determining cloud cells and
tracking these over time. Of interest will be: cell lifetime, cell lifecycle, number of cells,
statistics of cell spectrum and total cell mass flux (upwards and downwards). Another facet of the
study could involve looking at how these statistics are affected by performing the same analyses at
downscaled resolutions (e.g. $2 \Delta x$). 

Question 2 will build on question 1 by analysing how a parameterized model compares to a CRM when
the CRM is downscaled to the same resolution as the parameterized model. In particular, whether the
parameterized mass-flux matches the diagnosed mass-flux from the CRM, and how the flow in the
parameterized model matches the downscaled CRM flow. The latter could be done by comparing the
spectral decomposition of the respective flows.

In question 3 I will see how the overuse of parameterization impacts on a CRM resolution model. The
aim will be to shed some light on the way the parameterization interacts with the model dynamics
when both are effectively modelling the same phenomenon.

\subsection*{Hypotheses to be tested:}

\begin{itemize}
    \item \textbf{Energy balance:} Total surface heat flux should balance OLR.
    \item \textbf{Moisture balance:} Latent heat flux should balance precipitation.
    \item \textbf{Convergence:} Whether cell statistics converge when going to higher resolutions.
\end{itemize}

\subsection*{Model setup and runs}

The model used will be the Idealized UM (IdUM), version 10.3(4?). The model will be run with
interactive radiation, over a sea held at around \SI{300}{K}, with bi-periodic boundary conditions,
over a domain of \SI{60}{km} x \SI{60}{km}. Models will be run until a steady-state is reached, and
then for an additional time period to allow gathering of statistics (how long? considerations?).
High resolution runs without convective parameterization will be run at $\Delta x = $ \SI{200}{m},
\SI{300}{m} and \SI{400}{m} to allow for a comparison of resolutions and to check for convergence.
Parameterized runs will be done at $\Delta x = $ \SI{200}{m} and \SI{12}{km}, representing the
over-parameterized run and the sensibly parameterized run respectively. 

The Smagorinksy turbulence parameterization will be used throughout. (? check this.)

\subsection*{Output analysis}

Will need cell tracking tools, talk to Juwon Kim? Analysis done off-line using Python. 

\end{document}

